% Author Alfredo Sánchez Alberca (asalber@ceu.es)

\section{Formulas}
\begin{enumerate}[leftmargin=*]
\item The following data table contains the income of a company by quarters:
\begin{center} 
\begin{tabular}{rrrr}
\toprule 
1st Quarter & 2nd quarter & 3rd quarter & 4th quarter\\

€480.000,00 &  €560.000,00 & €320.000,00 & €720.000,00\\
\bottomrule
\end{tabular}
\end{center}

Open the Excel file \url{http://aprendeconalf.es/office/excel/exercises/formulas/exercise-1.xlsx} and do the following
operations:

\begin{enumerate}
\item Use a formula to calculate the fixed commissions in range B5:E5. The amount of fixed commissions appears in cell B15.
\item Use a formula to calculate the fixed commissions in range B6:E6. The percentage of variable commissions appears in cell B16.
\item Use a formula to calculate the Earnings Before Taxes (EBT), subtracting commissions to income, in range B7:E7.
\item Use a formula to calculate taxes in range B9:E9. The percentage of taxes appears in cell B17.
\item Use a formula to calculate Profit After Taxes (PAT), subtracting taxes to EBT.
\item Use a formula to calculate the annual income, fixed commissions, variable commissions, EBT, taxes and PAT.  
\end{enumerate}


\item A company has had annual sales of € 1.200.000 in 2015.
The sales increase forecast for next years appears in the next table.
\begin{center} 
\begin{tabular}{rrrrr}
\toprule 
2016 & 2017 & 2018 & 2019 & 2020\\
10\% & 12\% & 14\% & 16\% & 18\%\\
\bottomrule
\end{tabular}
\end{center}
A 30\% of expenses is assumed every year.

Open the Excel file \url{http://aprendeconalf.es/office/excel/exercises/formulas/exercise-2.xlsx} and do the following
operations:

\begin{enumerate}
\item Use a formula to calculate the sales forecast for every year in cells C4:G4 according to the sales increase percentage of
cells C12:G12.
\item Use a formula to calculate the expenses for every year assuming the constant percentage over sales of cell C14.
\item Use a formula to calculate the profit for every year. 
\item Use a formula to calculate the average annual sales, expenses and profit for years from 2015 to 2020.
\end{enumerate}


\item A car company have dealerships in several cities. The next table shows the number of vehicles sold in the last month in
every dealership. 
\begin{center}
\begin{tabular}{lcccc}
\toprule
Vehicle & Madrid & Barcelona & Valencia & Sevilla\\
\midrule
Van & 5 & 4 & 2 & 1\\
Lorry & 3 & 3 & 1 & 1\\
Car & 10 & 10 & 8 & 12\\
Motorcycle & 30 & 25 & 40 & 20\\
\bottomrule
\end{tabular}
\end{center}

If van price is €$12,800$, lorry price is €$27,000$, car price is €$11,750$ and motorcycle price is €$4,200$, do the following
operations:
\begin{enumerate}
\item Create a new Excel workbook and enter the previous table with the vehicle sales in range A1:E5.
\item Enter the vehicle prices in range F2:F5 with the header ``Unit price'' in cell F1. 
\item Use a formula to calculate the total sales by vehicle type in range G2:G5 and enter the header ``Total by vehicle'' in cell
G1. In the formula you have to use references to the cells with the unit prices. 
\item Use a formula to calculate the total sales by cities in range B6:E6 and enter the header ``Total by city'' in the cell A6.
In the formula you have to use references to the cells with the unit prices. 
\item Use a formula to calculate the total sales in cell F7 and apply it a bold face font format. 
\item Save the workbook with name \textsh{car-dealerships.xlsx}. 
\end{enumerate}


\end{enumerate}