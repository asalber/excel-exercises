% Author Alfredo Sánchez Alberca (asalber@ceu.es)

\section{Formulas}
\begin{enumerate}[leftmargin=*,resume]
\item \label{ex-income-quarters}The following data table contains the income of a company by quarters:
\begin{center} 
\begin{tabular}{rrrr}
\toprule 
1st Quarter & 2nd quarter & 3rd quarter & 4th quarter\\

€480.000,00 &  €560.000,00 & €320.000,00 & €720.000,00\\
\bottomrule
\end{tabular}
\end{center}

Open the Excel workbook
\href{http://aprendeconalf.es/office/excel/exercises/formulas/income-by-quarters.xlsx}{\textsf{income-by-quarters.xlsx}}
and do the following operations:

\begin{enumerate}
\item Use a formula to calculate the fixed commissions in range B5:E5. The amount of fixed commissions appears in cell B15.
\item Use a formula to calculate the variable commissions in range B6:E6. The percentage of variable commissions appears
in cell B16.
\item Use a formula to calculate the Earnings Before Taxes (EBT), subtracting commissions to income, in range B7:E7.
\item Use a formula to calculate taxes in range B9:E9. The percentage of taxes appears in cell B17.
\item Use a formula to calculate Profit After Taxes (PAT), subtracting taxes to EBT.
\item Use a formula to calculate the annual income, fixed commissions, variable commissions, EBT, taxes and PAT. 
\item Save the workbook. 
\end{enumerate}


\item \label{ex-sales-forecast}A company has had annual sales of €1.200.000 in 2015.
The sales increase forecast for next years appears in the next table.
\begin{center} 
\begin{tabular}{rrrrr}
\toprule 
2016 & 2017 & 2018 & 2019 & 2020\\
10\% & 12\% & 14\% & 16\% & 18\%\\
\bottomrule
\end{tabular}
\end{center}
A 30\% of expenses is assumed every year.

Open the Excel workbook
\href{http://aprendeconalf.es/office/excel/exercises/formulas/sales-forecast.xlsx}{\textsf{sales-forecast.xlsx}} and do
the following operations:

\begin{enumerate}
\item Use a formula to calculate the sales forecast for every year in cells C4:G4 according to the sales increase percentage of
cells C12:G12.
\item Use a formula to calculate the expenses for every year assuming the constant percentage over sales of cell C14.
\item Use a formula to calculate the profit for every year. 
\item Use a formula to calculate the average annual sales, expenses and profit for years from 2015 to 2020.
\item Save the workbook.
\end{enumerate}

\item A company has done several works in the last month. The number of worked hours, the materials expenses and the
amount budgeted appear in the table below.

\begin{center}
\begin{tabular}{lrrr}
\toprule
Work & Worked hours & Materials & Budget \\
Sevilla	  & 450	& €550,000 & €950,000\\ 
Barcelona &	275	& €375,000 & €625,000\\ 
Valencia  & 300 & €450,000 & €750,000\\ 
Madrid    & 725	& €600,000 & €1,050,000\\ 
\bottomrule
\end{tabular}
\end{center}

Open the Excel workbook
\href{http://aprendeconalf.es/office/excel/exercises/formulas/work-cost-analysis.xlsx}{\textsf{work-cost-analysis.xlsx}}
and do the following operations:

\begin{enumerate}
\item Assuming that the unit cost per hour is in cell B11, use a formula to calculate the total cost for every work in
range D3:D6.
\item Use a formula to calculate the difference between the total cost and the budgeted in range F3:F6.
\item Use a formula to calculate the minimum worked hours, material expenses and total cost in range B7:D7.
\item Use a formula to calculate the maximum worked hours, material expenses and total cost in range B8:D8.
\item Use a formula to calculate the average worked hours, material expenses and total cost in range B9:D9.
\item Apply a conditional formatting to the range A3:A6 to show the name of the work with the highest cost in red and
that with the lowest cost in green. 
\item Save the workbook. 
\end{enumerate}

\item The table below contains the total tax due and the amounts invested in donations and primary residence of the
income tax declaration of three contributors.
\begin{center}
\begin{tabular}{lrrr}
\toprule
 & Luis & Ramón & Ana\\
\midrule
Total tax due & €19.500 & €3.400 & €31.500\\
Donations & €1.500 & €800 & €1.200\\
Primary residence & €12.000 & €18.000 & €15.000\\
\bottomrule
\end{tabular}
\end{center}

Open the Excel workbook
\href{http://aprendeconalf.es/office/excel/exercises/formulas/income-tax-declaration.xlsx}{\textsf{income-tax-declaration.xlsx}}
and do the following operations:

\begin{enumerate}
\item Use a formula to calculate the tax deduction for donations in range B9-D9. According to IRPF law, tax deduction
for donations is 15\%.
\item Use a formula to calculate the tax deduction for primary residence in range B10-D10. According to IRPF law, tax deduction
for primary residence is 20\%.
\item Use a formula to calculate the total deductions in range B11-D11. 
\item Use a formula to calculate the total tax due minus deductions in range B13-D13. If it results a negative amount,
the cell value has to be 0. 
\item Save the workbook. 
\end{enumerate}


\item \label{ex-car-dealerships}A car company have dealerships in several cities. The table below shows the number of
vehicles sold in the last month in every dealership. 
\begin{center}
\begin{tabular}{lcccc}
\toprule
Vehicle & Madrid & Barcelona & Valencia & Sevilla\\
\midrule
Van & 5 & 4 & 2 & 1\\
Lorry & 3 & 3 & 1 & 1\\
Car & 10 & 10 & 8 & 12\\
Motorcycle & 30 & 25 & 40 & 20\\
\bottomrule
\end{tabular}
\end{center}

If van price is €$12,800$, lorry price is €$27,000$, car price is €$11,750$ and motorcycle price is €$4,200$, do the following
operations:
\begin{enumerate}
\item Create a new Excel workbook and enter the previous table with the vehicle sales in range A1:E5.
\item Enter the vehicle prices in range F2:F5 with the header ``Unit price'' in cell F1.
Give a name to every cell with a price. 
\item Use a formula to calculate the total sales by vehicle type in range G2:G5 and enter the header ``Total by vehicle'' in cell
G1. In the formula you have to use references to the cells with the unit prices. 
\item Use a formula to calculate the total sales by cities in range B6:E6 and enter the header ``Total by city'' in cell A6.
In the formula you have to use the named cells with the vehicles prices. 
\item Use a formula to calculate the total sales in cell F7 and apply it a bold face font format. 
\item Save the workbook in a file with name \textsf{car-dealerships.xlsx}. 
\end{enumerate}


\item The monthly sales of a company in the last quarter appears in the table below. 
\begin{center}
\begin{tabular}{ccc}
\toprule
October & November & December\\
€15.000 & €10.000 & €21.000\\
\bottomrule
\end{tabular}
\end{center}

Open the Excel workbook
\href{http://aprendeconalf.es/office/excel/exercises/formulas/last-quarter-balance.xlsx}{\textsf{last-quarter-balance.xlsx}}
and do the following operations:

\begin{enumerate}
\item Use a formula to calculate the salaries for every month in range B5:D5. The salaries have a fixed part that
appears in cell E5, a variable percentage of sales that appears in cell F5 and a plus percentage of sales if sales are greater than €20,000
that appears in cell G5.
\item Use a formula to calculate the rest of expenses for every month in range B6:D9. Each type of expenses have a fixed
part that appears in column E and a variable percentage of sales that appears in column F. 
\item Use a formula to calculate the total expenses for every month in range B10:D10. 
\item Use a formula to calculate the profits for every month in range B12:D12.
\item Use a formula to calculate the minimum expenditure of the quarter in cell B14.
\item Use a formula to calculate the maximum expenditure of the quarter in cell B15.
\item Use a formula to calculate the average expenditure of the quarter in cell B16.
\item Use a formula to calculate the number of expenses of the quarter over €1,500 in cell B14.
\item Apply a Data Bars conditional formatting to range B5:D9.
\item Save de workbook.
\end{enumerate}

\item An Internet provider wants to calculate how much have to paid their clients by the service.
The Excel workbook
\href{http://aprendeconalf.es/office/excel/exercises/formulas/internet-services.xlsx}{\textsf{internet-services.xlsx}}
contains the the starting and ending dates of several Internet users. 
Open the workbook and do the following operations:
\begin{enumerate}
\item Use a formula to calculate the number of days that users have been using Internet service. 
\item Use a formula to calculate the cost of the Internet service for every user.
Apply a discount to the final cost according to the table below:
\begin{center}
\begin{tabular}{lc}
\textbf{Days} & \textbf{Discount}\\
Less than 90 days & 0\%\\
Between 90 and 179 days & 5\%\\
Between 180 and 359 days & 10\%\\
360 days or more & 20\%\\
\end{tabular}
\end{center}  
Use references to cell B13 and range A16:B18.
\item Use a formula to calculate the total number of days in cell D11 and the total cost in cell E11. 
\item Use conditional formatting to apply to range D3:D10 a red font colour if number days is greater than or equal to
360, a yellow colour if number of days is between  180 and 350 and a green colour if number of days is between 90 and
179. 
\item Save de workbook. 
\end{enumerate}

\item \label{lemmonade}A lemonade shop wants to calculate its profits during the summer. 
The Excel
workbook\href{http://aprendeconalf.es/office/excel/exercises/formulas/payroll.xlsx}{\textsf{payroll.xlsx}} contains the
sales and cost of a lemonade shop from Junte to September. 
Open the workbook and do the following operations:
\begin{enumerate}
\item Use a formula to calculate the sales for every month in range C3:C6.
Use a reference to the unit price in cell B10. 
\item Use a formula to calculate the variable cost per unit in cell E15. 
\item Use a formula to calculate the variable cost for every month in range D3:D6.
Use a reference to the variable cost per unit in cell E15. 
\item Use a formula to calculate the total fixed cost in cell B16.
\item Use a formula to calculate the monthly apportionment of the fixed cost in cell B17.  
Divide the total fixed cost by the number of months.  
\item Use a formula to calculate the total cost (the sum of the variable cost and the fixed cost) for every month in
range E3:E6.
Use a reference to the monthly apportionment of the fixed cost in cell B17. 
\item Use a formula to calculate the profit for every month in cell F3:F6. 
\item Use a formula to calculate the total units sold, total sales, total variable cost, total cost, and total profit. 
\item Format the table like the one in the solution worksheet. 
\item Save the workbook. 
\end{enumerate}


\item The CEU academy wants to elaborate the payroll for the last month. 
The Excel workbook
\href{http://aprendeconalf.es/office/excel/exercises/formulas/payroll.xlsx}{\textsf{payroll.xlsx}} has a
worksheet with the basic salary of the CEU academy employees. 
Open the workbook and do the following operations:
\begin{enumerate}
\item Use a formula to calculate the overtime salary of every employee in range B7:F7. Use references to the extra hours
of each employee are in the range B21:F21 and the remuneration of an extra hour in cell B24.
\item Use a formula to calculate the commissions of every employee in range B8:F8.
The commissions are calculated applying a percentage on the basic salary.
Use references to the commission percentage of each employee in the range B22:F22.
\item Use a formula to calculate the antiquity bonus for every employee. An employee has right to the bonus if he stated
to work before the date of cell B19. The amount of the antiquity bonus is in cell B20.
\item Use a formula to calculate the gross salary for every employee in range B10:F10.
\item Use a formula to calculate the social security discount for every employee in range B12:F12.
The social security discounts are calculated applying the percentage of cell B25 on the gross salary. 
\item Use a formula to calculate the IRPF taxes for every employee in range B13:F13. 
The IRPF taxes are calculated applying a percentage on the gross salary.
Use references to the IRPF tax percentage of each employee in the range B26:F26.
\item Use a formula to calculate the net salary for every employee in range B15:F15.
\item Save the workbook.
\end{enumerate}

\item \label{ex-laboratory-suppliers}A laboratory has several suppliers of a product with different unit prices and
discounts.
The Excel workbook
\href{http://aprendeconalf.es/office/excel/exercises/formulas/laboratory-suppliers.xlsx}{\textsf{laboratory-suppliers.xlsx}}
has a worksheet with the supplied units in the first quarter of this year by months. 
Open the workbook and do the following operations: 
\begin{enumerate}
\item Use a formula to calculate the cost of supplied units by every supplier each month in range E10:G12.  
Apply the unit price of every supplier that appears in range B3:B5.
Also apply the discount of every supplier that appears in range C3:C5 if the supplied units are greater than or equal to
the minimum units to apply the discount that appear in range D3:D5. 
\item Use a formula to calculate the total cost of supplied units for every supplier in range H10:H12. 
\item Use a formula to calculate the total supplied units and the total cost of supplied units by months in range
B13:G13.
\item Use a formula to calculate the grand total cots in cell H13. 
\item Use a formula to show the month with the highest cost in cell D15. 
\item Use a formula to calculate the average monthly cost by suppliers in range I10:I12. 
\item Every supplier will apply a bonus discount the next quarter if the laboratory has an average monthly purchases
over a minimum specified in range E3:E5 and if the supplied units all the months are greater than or equal to the
minimum units specified in range F3:F5.
Use a formula to calculate if every supplier will apply a bonus discount next quarter in range J10:J12.
The formula should return ``YES'' or ``NO'' if the supplier will apply the bonus or not respectively. 
\item Save the workbook.   
\end{enumerate} 


\item \label{ex-salary-sellers}The table below contains the sales and the salaries of several sellers in a company:
\begin{center}
\begin{tabular}{lrrrr}
\toprule
& & \multicolumn{3}{c}{Sales}\\
\cline{3-5}
Seller & Salary & January & February & Mars \\
\midrule
López & €1,500 & €5,340 & €5,500 & €4,970\\
Merino & €2,100 & €3,560 & €4,525 & €2,850\\
Pastor & €700  & €2,850 & €2,450 & €1,850\\
Ramirez & €1,800 & €6,250 & €5,100 & €4,940\\
Zamora & €1,100 & €5,800 & €4,500 & €6,500\\
\bottomrule
\end{tabular}
\end{center}

According to the policy of the company, sellers with an average ratio between sales and salary greater than 5 will
receive a 5\% salary increase and sellers with an average ratio lower than 3 will be fired. 

Do the following operations:
\begin{enumerate}
\item Crate an Excel workbook an enter the previous table in range a A1:E7. Observe that the cells of range C1:E1 have
to be merged and centered. 
\item Use a formula to calculate the ratio between sales and salary (RSS) for every month and seller on range F3:H7.
Write appropriate headers on cells F1:H2.  
\item Use a formula to calculate the average RSS for every seller on range I3:I7.
Write an appropriate header on cell I2. 
\item Use a formula to write a message with the action to take according to the average RSS on range J3:J7.
The message should be ``Increase salary'' if RSS$>$5, ``Fire seller'' if RSS$<$5 and ``Nothing'' in any other case.
Write an appropriate header on cell J2. 
\item Use a formula to write a message with the new salary according to the action on column J.
If the action is ``Increase salary'' increase the salary a 5\% and if the action is ``Fire seller'' the new salary will
be 0. 
\item Apply an Icon Set conditional formatting to RSS in range F3:H7. Use a green icon if RSS$>$5, a yellow icon if
3$\leq$RSS$\leq$5 and a red icon if RSS$<3$. Use references to cells C10 and C11 in the conditions. 
\item Save the workbook in a file with name \textsf{salary-sellers.xlsx}. 
\end{enumerate}

\item \label{ex-quarter-payroll}The Excel workbook
\href{http://aprendeconalf.es/office/excel/exercises/formulas/quarter-payroll.xlsx}{\textsf{quarter-payroll.xlsx}}
contains the salaries of the employees of a company for the first quarter of this year.
The salaries of each month is in a distinct worksheet. 
Open the workbook and do the following operations:
\begin{enumerate}
\item In the January worksheet, use a formula to calculate the commission for every employee. 
The commissions are calculated applying the appropriate percentage of the Commissions and taxes worksheet to the sales. 
Calculate the commissions for the other months copying and pasting this formula. 
\item In the January worksheet, use a formula to calculate the gross salary for every employee. 
The gross salary is the sum of the basic salary in the salaries worksheet and the commissions. 
Calculate the gross salaries for the other months copying and pasting this formula. 
\item In the January worksheet, use a formula to calculate the IRPF for every employee. 
The IRPF taxes are calculated applying the appropriate percentage of the Commissions and taxes worksheet to the gross
salary.
Calculate the IRPF for the other months copying and pasting this formula. 
\item In the January worksheet, use a formula to calculate the social security discount for every employee. 
The social security discounts are calculated applying the appropriate percentage of the Commissions and taxes worksheet
to the gross salary.
Calculate the social security discounts for the other months copying and pasting this formula. 
\item In the January worksheet, use a formula to calculate the net salary for every employee. 
The net salaries are calculated subtracting IRPF and social security discounts form gross salaries.
Calculate the net salaries for the other months copying and pasting this formula.
\item In worksheet 1st quarter, use a formula to calculate the average monthly sales for every employee in range
B3:B7.
\item In worksheet 1st quarter, use a formula to calculate the incentive for every employee in range C3:C7.
The incentives are calculated applying the appropriate percentage of the Commissions and taxes worksheet
to the average monthly sales. 
\item In worksheet 1st quarter, use a formula to calculate the total gross salary of the first quarter for
every employee in range D3:D7.
\item In worksheet 1st quarter, use a formula to calculate the total gross salary of the first quarter for
every employee in range D3:D7.
\item In worksheet 1st quarter, use a formula to calculate the total IRPF of the first quarter for
every employee in range D3:D7.
\item In worksheet 1st quarter, use a formula to calculate the total social security discount of the first quarter for
every employee in range D3:D7.
\item Save the workbook. 
\end{enumerate}

\item \label{ex-autocarsa}The cars company AutocarSA has offices in Madrid, Barcelona, Sevilla and Valencia. 
The Excel workbook
\href{http://aprendeconalf.es/office/excel/exercises/formulas/autocarsa.xlsx}{\textsf{autocarsa.xlsx}} has a worksheet
with the vehicles sold in every quarter for every city. Also it has a worksheet with prices of every vehicle and another
worksheet with the taxes of every vehicle. 
Open the workbook and do the following operations:

\begin{enumerate}
\item In de Madrid worksheet, use a formula to calculate the amount sold (in €) for each type of vehicle in range F3:F6.
Use references to the prices worksheet. 
Copy and paste the formula in the worksheets of the other cities. 
\item In de Madrid worksheet, use a formula to calculate the registration taxes for each type of vehicle in range G3:G6.
Use references to the taxes worksheet. 
Copy and paste the formula in the worksheets of the other cities.
\item In de Madrid worksheet, use a formula to calculate the VAT for each type of vehicle in range H3:H6.
Use references to the taxes worksheet. 
Copy and paste the formula in the worksheets of the other cities.
\item In de Madrid worksheet, use a formula to calculate the amount sold (in €) for every quarter in range
B7:E7.
Use references to the prices worksheet. 
Copy and paste the formula in the worksheets of the other cities.
\item In de Madrid worksheet, use a formula to calculate the registration taxes for every quarter in range
B8:E8.
Use references to the taxes worksheet. 
Copy and paste the formula in the worksheets of the other cities.
\item In de Madrid worksheet, use a formula to calculate the VAT for every quarter in range
B9:E9.
Use references to the taxes worksheet. 
Copy and paste the formula in the worksheets of the other cities.
\item Use a formula to calculate the total amount in cell F7.
Copy and paste the formula in the worksheets of the other cities.  
\item Use a formula to calculate the total registration taxes in cell G8.
Copy and paste the formula in the worksheets of the other cities.  
\item Use a formula to calculate the total VAT in cell H9.
Copy and paste the formula in the worksheets of the other cities. 
\item Use a formula to calculate the grand total in cell I10.
Copy and paste the formula in the worksheets of the other cities.
\item Copy and paste the sales for every vehicle from the Madrid worksheet to the appropriate range in the Resume
worksheet.
Do the same for the other cities.  
\item Copy and paste the amount, registration taxes, VAT and total form the Madrid worksheet to the appropriate
range in the Resume worksheet. 
Do the same for the other cities. 
\item Use a formula to calculate the totals sales for every vehicle type in all the cities in range F3:F6 of the Resume
worksheet. 
\item Use a formula to calculate the total amounts, registrations taxes, VAT for all the cities in range F7:F9 of the
Resume worksheet. 
\item Use a formula to calculate the grand total for all the cities in cell F10 of the Resume worksheet. 
\end{enumerate} 

\item Use Excel to create the \href{https://en.wikipedia.org/wiki/Battleship_(game)}{battleship game}. 
The Excel workbook
\href{http://aprendeconalf.es/office/excel/exercises/formulas/laboratory-suppliers.xlsx}{\textsf{laboratory-suppliers.xlsx}}
contains a template of the game with two worksheet, one for the defense and one the attack. 
The defense worksheet contains a board with several boats that are represented with the symbol ``X''. 
The attack worksheet contains two boars, one in range A2:J11 where the player will shot writing a ``X'' in the chosen
cell, and the other in range L2:U11 where the player will see the result of its shots. 
The shot hits if the player writes an ``X'' in the same place where there is an ``X'' in the defense worksheet;
otherwise the shot fails. 
The player have 50 shots available.
If he destroy all the boats (hit all the ``X" in the defense board) before to spend the 50 shots, he wins the battle;
otherwise it loses. 
Open the workbook and do the following operations:
\begin{enumerate}
\item Use a formula to show the symbol ``O'' if the shot fails and the symbol ``+'' if the shot hits in range L2:U11.
\item Apply a conditional formatting to range L2:U11 to show a blue background if the cell contains the symbol ``O'' and
a red background if the cell contains a symbol ``+''.
\item Use a formula to show the number of ``X'' in the defense board in cell X3. 
\item Use a formula to show the number of spent shots in cell X4.
\item Use a formula to show the number of hits in cell X5.  
\item Use a formula to show the text ``Win'' or ``Lose'' in cell X6 according to the result of the game. 
\item Save the workbook.
\end{enumerate}

\end{enumerate}