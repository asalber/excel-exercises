% Author Alfredo Sánchez Alberca (asalber@ceu.es)

\section{Formulas}
\begin{enumerate}[leftmargin=*,resume]
\item \label{income-quarters}The following data table contains the income of a company by quarters:
\begin{center} 
\begin{tabular}{rrrr}
\toprule 
1st Quarter & 2nd quarter & 3rd quarter & 4th quarter\\

€480.000,00 &  €560.000,00 & €320.000,00 & €720.000,00\\
\bottomrule
\end{tabular}
\end{center}

Open the Excel file
\href{http://aprendeconalf.es/office/excel/exercises/formulas/income-by-quarters.xlsx}{\textsf{income-by-quarters.xlsx}}
and do the following operations:

\begin{enumerate}
\item Use a formula to calculate the fixed commissions in range B5:E5. The amount of fixed commissions appears in cell B15.
\item Use a formula to calculate the variable commissions in range B6:E6. The percentage of variable commissions appears
in cell B16.
\item Use a formula to calculate the Earnings Before Taxes (EBT), subtracting commissions to income, in range B7:E7.
\item Use a formula to calculate taxes in range B9:E9. The percentage of taxes appears in cell B17.
\item Use a formula to calculate Profit After Taxes (PAT), subtracting taxes to EBT.
\item Use a formula to calculate the annual income, fixed commissions, variable commissions, EBT, taxes and PAT. 
\item Save the workbook. 
\end{enumerate}


\item A company has had annual sales of €1.200.000 in 2015.
The sales increase forecast for next years appears in the next table.
\begin{center} 
\begin{tabular}{rrrrr}
\toprule 
2016 & 2017 & 2018 & 2019 & 2020\\
10\% & 12\% & 14\% & 16\% & 18\%\\
\bottomrule
\end{tabular}
\end{center}
A 30\% of expenses is assumed every year.

Open the Excel file
\href{http://aprendeconalf.es/office/excel/exercises/formulas/sales-forecast.xlsx}{\textsf{sales-forecast.xlsx}} and do
the following operations:

\begin{enumerate}
\item Use a formula to calculate the sales forecast for every year in cells C4:G4 according to the sales increase percentage of
cells C12:G12.
\item Use a formula to calculate the expenses for every year assuming the constant percentage over sales of cell C14.
\item Use a formula to calculate the profit for every year. 
\item Use a formula to calculate the average annual sales, expenses and profit for years from 2015 to 2020.
\item Save the workbook.
\end{enumerate}

\item A company has done several works in the last month. The number of worked hours, the materials expenses and the
amount budgeted appear in the table below.

\begin{center}
\begin{tabular}{lrrr}
\toprule
Work & Worked hours & Materials & Budget \\
Sevilla	  & 450	& €550,000 & €950,000\\ 
Barcelona &	275	& €375,000 & €625,000\\ 
Valencia  & 300 & €450,000 & €750,000\\ 
Madrid    & 725	& €600,000 & €1,050,000\\ 
\bottomrule
\end{tabular}
\end{center}

Open the Excel file
\href{http://aprendeconalf.es/office/excel/exercises/formulas/work-cost-analysis.xlsx}{\textsf{work-cost-analysis.xlsx}}
and do the following operations:

\begin{enumerate}
\item Assuming that the unit cost per hour is in cell B11, use a formula to calculate the total cost for every work in
range D3:D6.
\item Use a formula to calculate the difference between the total cost and the budgeted in range F3:F6.
\item Use a formula to calculate the minimum worked hours, material expenses and total cost in range B7:D7.
\item Use a formula to calculate the maximum worked hours, material expenses and total cost in range B8:D8.
\item Use a formula to calculate the average worked hours, material expenses and total cost in range B9:D9.
\item Apply a conditional formatting to the range A3:A6 to show the name of the work with the highest cost in red and
that with the lowest cost in green. 
\item Save the workbook. 
\end{enumerate}

\item The table below contains the total tax due and the amounts invested in donations and primary residence of the
income tax declaration of three contributors.
\begin{center}
\begin{tabular}{lrrr}
\toprule
 & Luis & Ramón & Ana\\
\midrule
Total tax due & €19.500 & €3.400 & €31.500\\
Donations & €1.500 & €800 & €1.200\\
Primary residence & €12.000 & €18.000 & €15.000\\
\bottomrule
\end{tabular}
\end{center}

Open the Excel file
\href{http://aprendeconalf.es/office/excel/exercises/formulas/income-tax-declaration.xlsx}{\textsf{income-tax-declaration.xlsx}}
and do the following operations:

\begin{enumerate}
\item Use a formula to calculate the tax deduction for donations in range B9-D9. According to IRPF law, tax deduction
for donations is 15\%.
\item Use a formula to calculate the tax deduction for primary residence in range B10-D10. According to IRPF law, tax deduction
for primary residence is 20\%.
\item Use a formula to calculate the total deductions in range B11-D11. 
\item Use a formula to calculate the total tax due minus deductions in range B13-D13. If it results a negative amount,
the cell value has to be 0. 
\item Save the workbook. 
\end{enumerate}


\item A car company have dealerships in several cities. The table below shows the number of vehicles sold in the
last month in every dealership. 
\begin{center}
\begin{tabular}{lcccc}
\toprule
Vehicle & Madrid & Barcelona & Valencia & Sevilla\\
\midrule
Van & 5 & 4 & 2 & 1\\
Lorry & 3 & 3 & 1 & 1\\
Car & 10 & 10 & 8 & 12\\
Motorcycle & 30 & 25 & 40 & 20\\
\bottomrule
\end{tabular}
\end{center}

If van price is €$12,800$, lorry price is €$27,000$, car price is €$11,750$ and motorcycle price is €$4,200$, do the following
operations:
\begin{enumerate}
\item Create a new Excel workbook and enter the previous table with the vehicle sales in range A1:E5.
\item Enter the vehicle prices in range F2:F5 with the header ``Unit price'' in cell F1. 
\item Use a formula to calculate the total sales by vehicle type in range G2:G5 and enter the header ``Total by vehicle'' in cell
G1. In the formula you have to use references to the cells with the unit prices. 
\item Use a formula to calculate the total sales by cities in range B6:E6 and enter the header ``Total by city'' in cell A6.
In the formula you have to use references to the cells with the unit prices. 
\item Use a formula to calculate the total sales in cell F7 and apply it a bold face font format. 
\item Save the workbook in a file with name \textsf{car-dealerships.xlsx}. 
\end{enumerate}


\item The monthly sales of a company in the last quarter appears in the table below. 
\begin{center}
\begin{tabular}{ccc}
\toprule
October & November & December\\
€15.000 & €10.000 & €21.000\\
\bottomrule
\end{tabular}
\end{center}

Open the Excel file
\href{http://aprendeconalf.es/office/excel/exercises/formulas/last-quarter-balance.xlsx}{\textsf{last-quarter-balance.xlsx}}
and do the following operations:

\begin{enumerate}
\item Use a formula to calculate the salaries for every month in range B5:D5. The salaries have a fixed part that
appears in cell E5, a variable percentage of sales that appears in cell F5 and a plus percentage of sales if sales are greater than €20,000
that appears in cell G5.
\item Use a formula to calculate the rest of expenses for every month in range B6:D9. Each type of expenses have a fixed
part that appears in column E and a variable percentage of sales that appears in column F. 
\item Use a formula to calculate the total expenses for every month in range B10:D10. 
\item Use a formula to calculate the profits for every month in range B12:D12.
\item Use a formula to calculate the minimum expenditure of the quarter in cell B14.
\item Use a formula to calculate the maximum expenditure of the quarter in cell B15.
\item Use a formula to calculate the average expenditure of the quarter in cell B16.
\item Use a formula to calculate the number of expenses of the quarter over €1,500 in cell B14.
\item Apply a Data Bars conditional formatting to range B5:D9.
\item Save de workbook.
\end{enumerate}


\item The table below contains the sales and the salaries of several sellers in a company:
\begin{center}
\begin{tabular}{lrrrr}
\toprule
& & \multicolumn{3}{c}{Sales}\\
\cline{3-5}
Seller & Salary & January & February & Mars \\
\midrule
López & €1,500 & €5,340 & €5,500 & €4,970\\
Merino & €2,100 & €3,560 & €4,525 & €2,850\\
Pastor & €700  & €2,850 & €2,450 & €1,850\\
Ramirez & €1,800 & €6,250 & €5,100 & €4,940\\
Zamora & €1,100 & €5,800 & €4,500 & €6,500\\
\bottomrule
\end{tabular}
\end{center}

According to the policy of the company, sellers with an average ratio between sales and salary greater than 5 will
receive a 5\% salary increase and sellers with an average ratio lower than 3 will be fired. 

Do the following operations:
\begin{enumerate}
\item Crate an Excel workbook an enter the previous table in range a A1:E7. Observe that the cells of range C1:E1 have
to be merged and centered. 
\item Use a formula to calculate the ratio between sales and salary (RSS) for every month and seller on range F3:H7.
Write appropriate headers on cells F1:H2.  
\item Use a formula to calculate the average RSS for every seller on range I3:I7.
Write an appropriate header on cell I2. 
\item Use a formula to write a message with the action to take according to the average RSS on range J3:J7.
The message should be ``Increase salary'' if RSS$>$5, ``Fire seller'' if RSS$<$5 and ``Nothing'' in any other case.
Write an appropriate header on cell J2. 
\item Use a formula to write a message with the new salary according to the action on column J.
If the action is ``Increase salary'' increase the salary a 5\% and if the action is ``Fire seller'' the new salary will
be 0. 
\item Apply an Icon Set conditional formatting to RSS in range F3:H7. Use a green icon if RSS$>$5, a yellow icon if
3$\leq$RSS$\leq$5 and a red icon if RSS$<3$. Use references to cells C10 and C11 in the conditions. 
\item Save the workbook in a file with name \textsf{salary-sellers.xlsx}. 
\end{enumerate}


\item The cars company AutocarSA has offices in Madrid, Barcelona, Sevilla and Valencia. 
The workbook
\href{http://aprendeconalf.es/office/excel/exercises/formulas/autocarsa.xlsx}{\textsf{autocarsa.xlsx}} has a worksheet
with the vehicles sold in every quarter for every city. Also it has a worksheet with prices of every vehicle and another
worksheet with the taxes of every vehicle. 
Open the workbook and do the following operations:

\begin{enumerate}
\item In de Madrid worksheet, use a formula to calculate the amount sold (in €) for each type of vehicle in range F3:F6.
Use references to the prices worksheet. 
Copy and paste the formula in the worksheets of the other cities. 
\item In de Madrid worksheet, use a formula to calculate the registration taxes for each type of vehicle in range G3:G6.
Use references to the taxes worksheet. 
Copy and paste the formula in the worksheets of the other cities.
\item In de Madrid worksheet, use a formula to calculate the VAT for each type of vehicle in range H3:H6.
Use references to the taxes worksheet. 
Copy and paste the formula in the worksheets of the other cities.
\item In de Madrid worksheet, use a formula to calculate the amount sold (in €) for every quarter in range
B7:E7.
Use references to the prices worksheet. 
Copy and paste the formula in the worksheets of the other cities.
\item In de Madrid worksheet, use a formula to calculate the registration taxes for every quarter in range
B8:E8.
Use references to the taxes worksheet. 
Copy and paste the formula in the worksheets of the other cities.
\item In de Madrid worksheet, use a formula to calculate the VAT for every quarter in range
B9:E9.
Use references to the taxes worksheet. 
Copy and paste the formula in the worksheets of the other cities.
\item Use a formula to calculate the total amount in cell F7.
Copy and paste the formula in the worksheets of the other cities.  
\item Use a formula to calculate the total registration taxes in cell G8.
Copy and paste the formula in the worksheets of the other cities.  
\item Use a formula to calculate the total VAT in cell H9.
Copy and paste the formula in the worksheets of the other cities. 
\item Use a formula to calculate the grand total in cell I10.
Copy and paste the formula in the worksheets of the other cities.
\item Copy and paste the sales for every vehicle from the Madrid worksheet to the appropriate range in the Resume
worksheet.
Do the same for the other cities.  
\item Copy and paste the amount, registration taxes, VAT and total form the Madrid worksheet to the appropriate
range in the Resume worksheet. 
Do the same for the other cities. 
\item Use a formula to calculate the totals sales for every vehicle type in all the cities in range F3:F6 of the Resume
worksheet. 
\item Use a formula to calculate the total amounts, registrations taxes, VAT for all the cities in range F7:F9 of the
Resume worksheet. 
\item Use a formula to calculate the grand total for all the cities in cell F10 of the Resume worksheet. 
\end{enumerate} 


\end{enumerate}