% Author Alfredo Sánchez Alberca (asalber@ceu.es)

\section{Dabase management}
\begin{enumerate}[leftmargin=*]
\item The workbook
\href{http://aprendeconalf.es/office/excel/exercises/formulas/billing-database.xlsx}{\textsf{billing-database.xlsx}}
has a database with the billing data of a computers company.
The database fields are city, the shop, the department, the employee, the type of sales or work, the quantity and the
amount billed. 
Open the workbook and do the following operations:
\begin{enumerate}
\item Format the datalist as a table.
\item Sort data alphabetically by city, then by shop, then by department and finally by employee.
\item Summarize the data list giving the subtotaling of de amount billed by cities. 
\item Summarize the data list giving the average quantity by departments. 
\item Create a pivot table and a column pivot chart with the following summaries: 
\begin{enumerate}
\item The total amount billed by cities.  
\item The total amount billed by cities, disaggregated by shops in rows, and by departments in columns.  
\item The total quantity by the type of product. 
\item The total quantity by the type of product in rows and by the employee in columns.  
\item The total quantity by city, disaggregated by type of product. 
\item The total amount billed by shop, disaggregated by employee.
\item The average quantity by department. 
\item The average quantity by employees of Sevilla. 
\item The total amount billed by the sales department in Gran Vía and Triana shops.    
\end{enumerate}
\item Filter the data list to show the only data of the sales department.
\item Filter the data list to show the records of employees whose name stats with letter A and have with an amount
billed over the average.
\item Calculate in cell J2 the number of employees with laptop sales over € 18,000.  
\item Calculate in cell J5 the sales of Barcelona employees with a quantity over 20 units. 
\item Calculate in cell J8 the employee with the maximum sales in the sales department.

\end{enumerate}


\end{enumerate}