% Author Alfredo Sánchez Alberca (asalber@ceu.es)

\section{Database management}
\begin{enumerate}[leftmargin=*,resume]
\item The workbook
\href{http://aprendeconalf.es/office/excel/exercises/databases/billing-database.xlsx}{\textsf{billing-database.xlsx}}
contains a database with the billing data of a computers company.
The database fields are city, the shop, the department, the employee, the type of sales or work, the quantity and the
amount billed. 
Open the workbook and do the following operations:
\begin{enumerate}
\item Format the data list as a table.
\item Sort data alphabetically by city, then by shop, then by department and finally by employee.
\item Summarize the data list giving the subtotaling of de amount billed by cities. 
\item Summarize the data list giving the average quantity by departments. 
\item Create a pivot table and a column pivot chart with the following summaries: 
\begin{enumerate}
\item The total amount billed by cities.  
\item The total amount billed by cities, disaggregated by shops in rows, and by departments in columns.  
\item The total quantity by the type of product. 
\item The total quantity by the type of product in rows and by the employee in columns.  
\item The total quantity by city, disaggregated by type of product. 
\item The total amount billed by shop, disaggregated by employee.
\item The average quantity by department. 
\item The average quantity by employees of Sevilla. 
\item The total amount billed by the sales department in Gran Vía and Triana shops.    
\end{enumerate}
\item Filter the data list to show the only data of the sales department.
\item Filter the data list to show the records of employees whose name stats with letter A and have an amount
billed over the average.
\item Calculate in cell J2 the number of employees with laptop sales over € 18,000.  
\item Calculate in cell J5 the sales of Barcelona employees with a quantity over 20 units. 
\item Calculate in cell J8 the employee with the maximum sales in the sales department.
\end{enumerate}


\item The workbook
\href{http://aprendeconalf.es/office/excel/exercises/databases/workers-database.xlsx}{\textsf{workers-database.xlsx}}
contains a database with data about 6366 workers of 5 EEUU states.
The database fields are sex, race, state, years of education, study level and the yearly earnings.
Open the workbook and do the following operations:
\begin{enumerate}
\item Format the data list as a table.
\item Sort data alphabetically by state, study level, then yearly earnings.
\item Filter the data list to get the workers with more than 16 education years and yearly earnings under \$ $20,000$.   
\item Summarize the data list giving the subtotaling of de yearly earnings by states. 
\item Summarize the data list giving the average yearly earnings by sex. 
\item Create a pivot table and a column pivot chart with the following summaries: 
\begin{enumerate}
\item The average yearly earnings and the count yearly earnings by states. 
\item The average yearly earnings and the count yearly earnings by study levels.
\item The average yearly earnings by state in rows and by sex in columns.
\item The average yearly earnings and education years by state in rows an by sex in columns.
\item The average yearly earnings by sex and race for workers with a master degree or a doctorate. 
\item The count yearly earnings by yearly earnings intervals of with \$ $50,000$.
\end{enumerate}
\end{enumerate}


\item The workbook
\href{http://aprendeconalf.es/office/excel/exercises/databases/spasnish-debt.xlsx}{\textsf{spanish-debt.xlsx}}
contains a database with data about the population, the GDP and the debt of the spanish economy in the last years. 
Open the workbook and do the following operations:
\begin{enumerate}
\item Use a formula to calculate the debt as percentage of the GDP in range E2:E92 of the Debt worksheet. Use the
VLOOKUP function.
\item Use a formula to calculate the debt per capita in € in range F2:F92 of the Debt worksheet. 
\item Create a pivot table and a line pivot chart with the following summaries: 
\begin{enumerate}
\item The total debt of the whole spanish economy by years. 
\item The debt by organism in rows and by years in columns. 
\item The debt (as a \%GDP) by type in rows and by years in columns.
\item The private (as a \%GDP) debt by organism in rows and by years in columns. 
\item The public debt (as a \%GDP) by organism in rows and by years in columns.
\item The public debt per capita by organism in rows and by years in columns.
\end{enumerate}
\item Create a pivot table and a sector pivot chart to show the following:
\begin{enumerate}
\item The 2007 debt by organism. 
\item The 2012 debt by organism.
\item The 2007 debt by type. 
\item The 2012 debt by type. 
\end{enumerate}
\item Create a pivot table and a column pivot chart to show the average debt (as \%GDP) by organism. 
\end{enumerate}


\end{enumerate}
