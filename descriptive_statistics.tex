% Author Alfredo Sánchez Alberca (asalber@ceu.es)

\section{Descriptive Statistics}
\begin{enumerate}[leftmargin=*,resume]
\item A poll on voting intention of citizens for the next election has surveyed 400 people of the three million people with right
to vote who live in a city. Identify:
\begin{enumerate}
\item The study population and its size ($N$).
\item The sample and its size ($n$).
\item The individual.
\item The studied variable and its scale.
\end{enumerate}


\item It is intended to conduct a study on the number of women looking for a job certain autonomous region. It is asked:
\begin{enumerate}
\item Describe the population and the sample to be studied.
\item Identify the individual or elementary unit in the study.
\item Define the variable to be studied and classify it correctly.
\end{enumerate}


\item The manager of a publishing house aims to determine the areas of knowledge of the books with greater acceptance in the
market.
Due to the large number of books for sale, he only study 15\% of all books published. 
Answer the following questions:
\begin{enumerate}
\item What is the study population?
\item What is the sample selected?
\item What is the individual?
\item What is the variable variable to study? Classify it.
\end{enumerate}

\item The director of a small company has conducted a survey among his workers that asked for the number of extra hours that they
need every week. 
Identify: 
\begin{enumerate}
\item The study population.
\item The selected sample.
\item The studied individual.
\item The studied variable and its scale.
\end{enumerate}

\item Classify, giving a reasoned answer, the following variables according to their scale:
\begin{enumerate}
\item Number of inhabitants per square Kilometre. 
\item Types of canned food products.
\item Family income of a group of families. 
\item Number of fruits per tree. 
\item Level of education.
\item The start-number of a runner. 
\item The temperature in degrees Celsius. 
\item The job function in a department of a company. 
\end{enumerate}

\item Give 3 examples of each type of economics variables according to their scale. 

\item Classify the following variables according to their categories: 
\begin{enumerate}
\item Grade of an exam (SS, AP, NT, SB, MH).
\item Category of a hotel ($\star$, $\star\star$, $\star\star\star$, $\star\star\star\star$, $\star\star\star\star\star$).
\item Amount of money, in €, that a young people spends on leisure ($0-10$, $10-30$, $30-60$).
\item Price of a bus ticket in € (the exact amount).
\item Surface in $m^2$ of a house ($0-50$, $50-80$, $80-100$, $110-200$, $200-$).
\end{enumerate}

\item Transform the variable that measures the surface of a house, in $m^2$, in a variable with ordinal scale, specifying their categories. 

\item Transform the variable that measures the mark in an exam in:
\begin{enumerate}
\item A variable with ordinal scale, specifying their categories and the order.
\item A variable with nominal scale, specifying their categories.
\end{enumerate}


\item The currencies used in a sample of financial transactions are shown below
\begin{center}
\begin{tabular}{llllllllll}
Euro & Pound & Euro & Dollar & Dollar & Dollar & Yen & Yuan & Yen & Euro \\
Yen & Yuan & Pound & Dollar & Dollar & Euro & Euro & Yen & Pound & Pound \\
Dollar & Euro & Dollar & Yuan & Dollar & Yuan & Dollar & Euro & Dollar & Dollar
\end{tabular} 
\end{center}
\begin{enumerate}
\item Construct a frequency table of the currencies.
\item Create a pie chart for the relative frequency of currencies. 
\item From the frequency table distribution, answer the following questions:
\begin{enumerate}
\item How many transactions contains the sample?
\item How many different currencies there are in the sample?
\item Which is the more common currency in the sample of transactions?
\item Which is the less common currency in the sample of transactions?
\end{enumerate}
\end{enumerate}


\item A sample of 16 stock shares has been evaluated according to the investment risk using the following scale: A= No
risk, B= Low risk, C= Moderate risk, D= High risk and E= Extreme risk.
The category for every stock share is shown below
\begin{center}
\begin{tabular}{cccccccc}
B & C & B & A & C & D & A & E\\
C & D & D & A & C & E & C & B 
\end{tabular} 
\end{center}
\begin{enumerate}
\item Construct a frequency table of the risk categories.
\item Create bar chart for the absolute frequency of risk.
\item Create a pie chart for the relative frequency of risk. 
\item From the frequency table distribution, answer the following questions:
\begin{enumerate}
\item Which is the more common risk category in the sample?
\item What percentage of shares have a moderate risk?
\item What percentage of shares have a moderate or lower risk?
\item What percentage of shares have an high or higher risk?
\end{enumerate}
\end{enumerate}


\item The number of subjects coursed by a sample of 20 students in a course year are
\[
\begin{array}{cccccccccc}
5 & 6 & 7 & 7 & 5 & 3 & 4 & 7 & 6 & 6\\
4 & 6 & 5 & 6 & 6 & 5 & 7 & 3 & 4 & 5 \\
\end{array}
\]
\begin{enumerate}
\item Construct a frequency table of subjects.
\item Create bar charts for all the type of frequencies. 
\item From the frequency table distribution, answer the following questions:
\begin{enumerate}
\item Which is the more common number of subjects coursed in a year?
\item How many students has coursed less than 5 subjects?
\item What percentage of students have coursed less than or equal 5 subjects?
\item What percentage of students have coursed more than 5 subjects?
\end{enumerate}
\end{enumerate}


\item\label{office-rental} The monthly rental (in €) per square metre for office space in the centre of Madrid are shown below
\[
\begin{array}{cccccccccc}
70 & 50 & 24 & 75 & 25 & 56 & 25 & 77 & 26 & 48 \\
50 & 78 & 21 & 30 & 40 & 65 & 75 & 24 & 55 & 50 \\
76 & 27 & 65 & 60 & 68 & 42 & 40 & 60 & 48 & 54
\end{array}
\]
The data are recorded in the Excel workbook \href{http://aprendeconalf.es/office/excel/exercises/descriptive-statistics/office-rental.xlsx}{\textsf{office-rental.xlsx}}.

\begin{enumerate}
\item Construct a frequency table of office rentals using classes of width € 10. 
\item Create a histogram for the absolute frequency.
\item Create a histogram for the cumulative absolute frequency. 
\item From the frequency distribution, answer the following questions:
\begin{enumerate}
\item What percentage of office space costs less than or equal to € $40/m^2$.
\item What percentage of office space costs less than or equal to € $60/m^2$.
\item What percentage of office space costs more than € $50/m^2$.
\item If a company that is looking to hire office space can pay between € $30/m^2$ and € $50/m^2$, how many buildings
can they consider?
\end{enumerate}
\end{enumerate}


\item\label{grades-sex} The workbook
\href{http://aprendeconalf.es/office/excel/exercises/descriptive-statistics/grades-sex.xlsx}{\textsf{grades-sex.xlsx}}
contains the grades in economics of 500 students. 
\begin{enumerate}
\item Construct a frequency table of grades using the classes 0-5, 5-7, 7-9, 9-10, and the corresponding absolute
frequency histogram. 
\item Construct two frequency tables of grades (one for men an other for women) using 10 classes of width 1. 
\item Create a overlapped histogram for relative frequency of grades, overlapping distributions of men an women.  
\item Create a stacked histogram for relative frequency of grades, stacking frequencies of men and women. 
\item Create relative frequency histograms of grades for men and women. 
Who has obtained better grades, men or women?  
\end{enumerate}


\item The table below contains the frequency of call durations (in minutes) of a sample of clients of a mobile phone
company.
\begin{center}
\begin{tabular}{cc}
\toprule
Duration & Calls\\
0-5 & 42\\
5-10 & 68\\
10-15 & 44\\
15-20 & 21\\
20-25 & 12\\
25-30 & 5\\
\bottomrule
\end{tabular}
\end{center}
\begin{enumerate}
\item Complete the frequency table of call durations. 
\item Create a histogram for the relative frequency.
\item Create a line chart or polygon for the relative frequency. 
\item Create a histogram for the cumulative relative frequency. 
\item Create a line chart or ogive for the cumulative relative frequency. 
\end{enumerate}


\item\label{i+d-investment} The data below shows the I+D investment (in thousand of euros) of a sample of pharmaceutical companies:
\[
\begin{array}{rrrrrrrrrr}
1,350 & 1,690 & 1,250 & 1,490 & 1,970 & 2,210 & 2,200 & 1,470 & 1,650 & 1,780 \\
2,120 & 1,840 & 1,950 & 1,950 & 2,180 & 1,390 & 2,120 & 2,300 & 1,590 & 1,480 \\
2,010 & 1,590 & 1,920 & 2,140 & 1,780 & 1,880 & 2,050 & 1,960 & 1,780 & 2,010 
\end{array}
\]
The data are recorded in the Excel workbook \href{http://aprendeconalf.es/office/excel/exercises/descriptive-statistics/i+d-investment.xlsx}{\textsf{i+d-investment.xlsx}}.

\begin{enumerate}
\item Construct the frequency distribution table grouping data in 5 classes with width 210, starting in 1250 and
finishing in 2300.
\item Create a histogram for the relative frequency and plot the corresponding polygon.  
Which shape has the histogram?
\item Create a histogram for the cumulative relative frequency. 
\item Compute the mean for the investment and interpret it. 
\item Which investment value represents the 50\% of the distribution?
\item How much invest a company in I+D usually?
\item How much invest the 70\% of the companies as much?
\item Which percentage of the companies invest less than € $1,880$?
\item Compute the quartiles for the investment and interpret them. 
\item Compute the inter-quartile range for the investment and interpret it. 
\item Create a box and whiskers plot for the investment and interpret it. 
\end{enumerate}


\item\label{household-savings} A survey has asked 30 Spanish families about their annual savings. 
The data are in the Excel workbook
\href{http://aprendeconalf.es/office/excel/exercises/descriptive-statistics/household-savings.xlsx}{\textsf{household-savings.xlsx}}
\begin{enumerate}
\item Construct the frequency distribution table grouping data. 
\item Create a histogram for the annual savings. 
\item Create an ogive for the annual savings. 
\item What is the average annual savings of the families?
\item Which percentage of families have annual savings under € $12,500$?
\item Compute and interpret the median. 
\item Which is the minimum annual savings that achieve the 63\% of the most thrifty families?
\end{enumerate}


\item The Excel workbook
\href{http://aprendeconalf.es/office/excel/exercises/databases/workers.xlsx}{\textsf{workers.xlsx}}
contains a database with data about 6366 workers of 5 EEUU states.
\begin{enumerate}
\item What is the average yearly earnings of workers?
\item What is the average yearly earnings by sex?
\item Compute the standard deviation for yearly earnings by sex.
Who has less variability in the yearly earnings, men or women?
\item What is the average yearly earnings by states?
In which state is more representative the mean?
\item How is the skewness of the education years distribution?
\item How is the kurtosis of the education years distribution?
\end{enumerate}


\item The Excel workbook
\href{http://aprendeconalf.es/office/excel/exercises/descriptive-statistics/phone-bills.xlsx}{\textsf{phone-bills.xlsx}}
contains the phone bills of a sample of clients of a mobile phone company. 

Compute the following statistics and interpret them:
\begin{enumerate}
\item Mean
\item Median
\item Mode
\item Quartiles
\item Percentile 65
\item Variance
\item Standard deviation
\item Coefficient of variation
\item Coefficient of skewness
\item Coefficient of kurtosis
\end{enumerate}

\item The Excel workbook \href{http://aprendeconalf.es/office/excel/exercises/descriptive-statistics/water-consumption.xlsx}{\textsf{water-consumption.xlsx}}
contains the volume of water used by a sample of households of a city. 

\begin{enumerate}
\item Compute and interpret the quartiles.
\item Create a box and whiskers chart. 
Are there outliers in the sample?
\item How is the skewness of the sample distribution?
\end{enumerate}

\item An investor wants to decide where to invest some money between two types of investments (A or B). In order to take the best decision the investor has taken a sample of annual rates of return from each type of investment. 
The data are in the Excel workbook \href{http://aprendeconalf.es/office/excel/exercises/descriptive-statistics/investment-returns.xlsx}{\textsf{investment-returns.xlsx}}
\begin{enumerate}
\item Compute the means for every type of investment and interpret it. 
\item Compute the standard deviation for every type of investment and interpret it.
\item Which type of investment has less variability?
\item Compute the standard score for every value in the sample.
Are there outliers in the sample?
\item Compute the quartiles and interpret them. 
\item Compute the interquartile range and interpret it. 
\item Create a box and whiskers chart with different boxes for every type of investment. 
\item What is the best type of investment? Why?
\end{enumerate}

\item Use the sample of exercise~\ref{office-rental} to compute:
\begin{enumerate}
\item The average rental per square metre. 
Is representative of the sample?
\item The third decile.
\item The skewness coefficient. Interpret it. 
\item The kurtosis coefficient. 
Interpret it.
Does this sample come from a normal population?
\end{enumerate}

\item Use the sample of exercise~\ref{household-savings} to compute:
\begin{enumerate}
\item The mean.
\item The standard deviation.
\item The coefficient of variation.
\item The standard scores.
Are there outliers in the sample?
\end{enumerate}

\item Use the sample of exercise~\ref{grades-sex} to answer the following questions:
\begin{enumerate}
\item Who has a higher mean, men or women?
What mean is more representative?
\item Create a box and whiskers chart with different boxes for every sex. 
\item What distribution is more symmetric, the men or the women distribution?
\item What distribution has a kurtosis more normal, the men or the women distribution?
\end{enumerate}

\end{enumerate}