% Author Alfredo Sánchez Alberca (asalber@ceu.es)

\section{Descriptive Statistics}
\begin{enumerate}[leftmargin=*,resume]
\item A poll on voting intention of citizens for the next election has surveyed 400 people of the three million people with right
to vote who live in a city. Identify:
\begin{enumerate}
\item The study population and its size ($N$).
\item The sample and its size ($n$).
\item The individual.
\item The studied variable and its scale.
\end{enumerate}


\item It is intended to conduct a study on the number of women looking for a job certain autonomous region. It is asked:
\begin{enumerate}
\item Describe the population and the sample to be studied.
\item Identify the individual or elementary unit in the study.
\item Define the variable to be studied and classify it correctly.
\end{enumerate}


\item The manager of a publishing house aims to determine the areas of knowledge of the books with greater acceptance in the
market.
Due to the large number of books for sale, he only study 15\% of all books published. 
Answer the following questions:
\begin{enumerate}
\item What is the study population?
\item What is the sample selected?
\item What is the individual?
\item What is the variable variable to study? Classify it.
\end{enumerate}

\item The director of a small company has conducted a survey among his workers that asked for the number of extra hours that they
need every week. 
Identify: 
\begin{enumerate}
\item The study population.
\item The selected sample.
\item The studied individual.
\item The studied variable and its scale.
\end{enumerate}

\item Classify, giving a reasoned answer, the following variables acording to their scale:
\begin{enumerate}
\item Number of inhabitants per square kilometer. 
\item Types of canned food products.
\item Family income of a group of families. 
\item Number of fruits per tree. 
\item Level of education.
\item The start-number of a runner. 
\item The temperature in degrees celsius. 
\item The job function in a department of a company. 
\end{enumerate}

\item Give 3 examples of each type of economics variables according to their scale. 

\item Classify the following variables according to their categories: 
\begin{enumerate}
\item Grade of an exam (SS, AP, NT, SB, MH).
\item Category of a hotel ($\star$, $\star\star$, $\star\star\star$, $\star\star\star\star$, $\star\star\star\star\star$).
\item Amount of money, in €, that a young people spends on leisure ($0-10$, $10-30$, $30-60$).
\item Price of a bus ticket in € (the exact amount).
\item Surface in $m^2$ of a house ($0-50$, $50-80$, $80-100$, $110-200$, $200-$).
\end{enumerate}

\item Transform the variable that measures the surface of a house, in $m^2$, in an variable with ordinal scale,
specifying their categories. 

\item The manager of an assembly plant in a car factory wants to study the productivity (number of assembled units per hour) of
their workers.
He measures the productivity taking note of the units leaving the conveyor belt of each worker. 
Classify the variable that measures the productivity.
Transform this variable in variables of interval, ordinal and nominal types, giving their categories in each case.   

\end{enumerate}