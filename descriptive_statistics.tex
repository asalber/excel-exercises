% Author Alfredo Sánchez Alberca (asalber@ceu.es)

\section{Descriptive Statistics}
\begin{enumerate}[leftmargin=*,resume]
\item A poll on voting intention of citizens for the next election has surveyed 400 people of the three million people with right
to vote who live in a city. Identify:
\begin{enumerate}
\item The study population and its size ($N$).
\item The sample and its size ($n$).
\item The individual.
\item The studied variable and its scale.
\end{enumerate}


\item It is intended to conduct a study on the number of women looking for a job certain autonomous region. It is asked:
\begin{enumerate}
\item Describe the population and the sample to be studied.
\item Identify the individual or elementary unit in the study.
\item Define the variable to be studied and classify it correctly.
\end{enumerate}


\item The manager of a publishing house aims to determine the areas of knowledge of the books with greater acceptance in the
market.
Due to the large number of books for sale, he only study 15\% of all books published. 
Answer the following questions:
\begin{enumerate}
\item What is the study population?
\item What is the sample selected?
\item What is the individual?
\item What is the variable variable to study? Classify it.
\end{enumerate}

\item The director of a small company has conducted a survey among his workers that asked for the number of extra hours that they
need every week. 
Identify: 
\begin{enumerate}
\item The study population.
\item The selected sample.
\item The studied individual.
\item The studied variable and its scale.
\end{enumerate}

\item Classify, giving a reasoned answer, the following variables according to their scale:
\begin{enumerate}
\item Number of inhabitants per square Kilometre. 
\item Types of canned food products.
\item Family income of a group of families. 
\item Number of fruits per tree. 
\item Level of education.
\item The start-number of a runner. 
\item The temperature in degrees Celsius. 
\item The job function in a department of a company. 
\end{enumerate}

\item Give 3 examples of each type of economics variables according to their scale. 

\item Classify the following variables according to their categories: 
\begin{enumerate}
\item Grade of an exam (SS, AP, NT, SB, MH).
\item Category of a hotel ($\star$, $\star\star$, $\star\star\star$, $\star\star\star\star$, $\star\star\star\star\star$).
\item Amount of money, in €, that a young people spends on leisure ($0-10$, $10-30$, $30-60$).
\item Price of a bus ticket in € (the exact amount).
\item Surface in $m^2$ of a house ($0-50$, $50-80$, $80-100$, $110-200$, $200-$).
\end{enumerate}

\item Transform the variable that measures the surface of a house, in $m^2$, in a variable with ordinal scale, specifying their categories. 

\item Transform the variable that measures the mark in an exam in:
\begin{enumerate}
\item A variable with ordinal scale, specifying their categories and the order.
\item A variable with nominal scale, specifying their categories.
\end{enumerate}

\item The data below shows the I+D investment (in thousand of euros) of a sample of pharmaceutical companies:
\[
\begin{array}{rrrrrrrrrr}
1,350 & 1,690 & 1,250 & 1,490 & 1,970 & 2,210 & 2,200 & 1,470 & 1,650 & 1,780 \\
2,120 & 1,840 & 1,950 & 1,950 & 2,180 & 1,390 & 2,120 & 2,300 & 1,590 & 1,480 \\
2,010 & 1,590 & 1,920 & 2,140 & 1,780 & 1,880 & 2,050 & 1,960 & 1,780 & 2,010 
\end{array}
\]

\begin{enumerate}
\item Construct the frequency distribution table grouping data in 5 classes with width 210, starting in 1250 and
finishing in 2300.
\item Create a histogram for the relative frequency and plot the corresponding polygon.  
Which shape has the histogram?
\item Create a histogram for the cumulative relative frequency. 
\item Compute the mean for the investment and interpret it. 
\item Which investment value represents the 50\% of the distribution?
\item How much invest a company in I+D usually?
\item How much invest the 70\% of the companies as much?
\item Which percentage of the companies invest less than € $1,880$?
\item Compute the quartiles for the investment and interpret them. 
\item Compute the inter-quartile range for the investment and interpret it. 
\item Create a box and whiskers plot for the investment and interpret it. 
\end{enumerate}


\item A survey has asked 30 Spanish families about their annual savings. 
The data are in the Excel workbook
\href{http://aprendeconalf.es/office/excel/exercises/databases/household-savings.xlsx}{\textsf{household-savings.xlsx}}

\begin{enumerate}
\item Construct the frequency distribution table grouping data. 
\item Create a histogram for the annual savings. 
\item Create an ogive for the annual savings. 
\item What is the average annual savings of the families?
\item Which percentage of families have annual savings under € $12,500$?
\item Compute and interpret the median. 
\item Which is the minimum annual savings that achieve the 63\% of the most thrifty families?
\end{enumerate}



\end{enumerate}