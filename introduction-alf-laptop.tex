% Author Alfredo Sánchez Alberca (asalber@ceu.es)

\section{Introduction to Excel}
\begin{enumerate}[leftmargin=*]
\item The following data table contains the population of several cities by years (in thousands of inhabitants):
\begin{center} 
\begin{tabular}{lrrrr}
\toprule 
City & 	2001 & 2006 & 2011 & 2014\\
\midrule
Madrid & 2957 & 3129 & 3265 & 3165\\
Barcelona & 1505 & 1606 & 1615 & 1602 \\ 
Valencia & 750 & 805 & 798 & 786\\
Zaragoza & 611 & 649 & 675 & 666\\
\bottomrule
\end{tabular}
\end{center}

It is asked:

\begin{enumerate}
\item Enter the population data in an Excel worksheet with the same table structure.
\item Enter the population data of year 1996 in a new column before the column of year 2001.
\begin{center} 
\begin{tabular}{lr}
\toprule 
City & 	1996\\
\midrule
Madrid & 2867\\
Barcelona & 1509\\
Valencia & 747\\
Zaragoza & 602\\
\bottomrule
\end{tabular}
\end{center}

\item Enter the population data of the Sevilla city in a new row below the Valencia row. 
\begin{center} 
\begin{tabular}{lrrrrr}
\toprule 
Year & 	1996 & 2001 & 2006 & 2011 & 2014\\
Sevilla & 697 & 684 & 704 & 703 & 697\\
\bottomrule
\end{tabular}
\end{center}

\item Save the workbook in a file with name \textsf{population.xlxs}.
\item Copy the row of Barcelona and paste it in row 10. 
\item Copy the column of year 2014 and paste it in column H. 
\item Copy the range with the population of Madrid and Barcelona in years 2001, 2006 and 2011 in range F8:H9. 
\item Save the modified workbook in another file with name \textsf{modified-population.xlxs}
\end{enumerate}


\item Open the Excel workbook
\href{http://aprendeconalf.es/office/excel/exercises/introduction/basic-edition.xlsx}{\textsf{basic-edition.xlsx}} and
do the following operations:
\begin{enumerate}
\item Enter the word Excel in the cell B8.
\item Enter the current year in cell C8.
\item Copy the content of cell A2 in cell C10.
\item Copy the content of range B8:C8 to range D12:E12.
\item Remove the content of cell A5.
\end{enumerate}

\item Open the Excel workbook
\href{http://aprendeconalf.es/office/excel/exercises/introduction/basic-autofill.xlsx}{\textsf{basic-autofill.xlsx}} and
do the following operations:
\begin{enumerate}
\item Replicate the content of cell  A6 to A12.
\item Auto fill the content of cells D6 to J6 with the days of the week.
\item Auto fill the content of cells B6 to B12 with the next dates to the date in B6.
\item Auto fill the content of cells C6 to C12 with the numbers of series that starts with numbers in cells  C6 and C7.
\end{enumerate}

\item \label{ex-basic-formatting}The Excel workbook 
\href{http://aprendeconalf.es/office/excel/exercises/introduction/basic-formatting.xlsx}{\textsf{basic-formatting.xlsx}}
contains the expenses of an academy for several months.
Open it and do the following operations:
\begin{enumerate}
\item Rename \textsf{Sheet1} as \textsf{Expenses}.
\item Insert a new row before row 1 and enter the text ``CEU Academy: 1st quarter expenses'' in cell A1.
\item Merge and center cells A1 to E1.
\item Format cell A1 with 18 pt boldface Arial font family and blue colour.
\item Increase the height of row 1 to 50 pt.
\item Align vertically text of cell A1 to the top.
\item Adjust the width of column A to the content of its cells.
\item Wrap text of cell E2.
\item Center content of cells A2:E2.
\item Format numeric cells to display values in currency format with 2 decimal places.
\item Format cells E3:E9 with boldface font.
\item Apply a thick top border to cells A2:E2.
\item Apply a thin top and thick bottom borders to cells A9:E9.
\item Apply a dark blue lighter 60\% colour background to cells A2:E2 and A9:E9.
\item Insert the text ``GRAND TOTAL'' in cell C12.
\item Copy value of cell E9 and paste it to cell E12.
\item Copy format of cell E9 to cell E12. 
\item Format cells C12:E12 with 14 pt font.
\item Save the file as a new file named \textsf{ceu-academy-expenses.xlsx}.
\end{enumerate}

\item\label{ex-invoice-template} Create an Excel worksheet with an invoice template like the one in the file
\href{http://aprendeconalf.es/office/excel/exercises/introduction/invoice.pdf}{\textsf{invoice.pdf}} and save it in a file with name \textsf{invoice.xlsx}.


\item The csv file
\href{http://aprendeconalf.es/office/excel/exercises/introduction/ibex-august-2015.csv}{\textsf{ibex-august-2015.csv}}
contains the IBEX values during August 2015.
Do the following operations. 
\begin{enumerate}
\item Import the csv file with Excel.
Observe that fields are separated by semicolons that have to be changed by commas in order to Excel recognize
the fields.\\
Hint: Use a plain text processor to change semicolons by commas.  
\item Insert a new row before row 1 and enter the text ``IBEX evolution August 2015'' in cell A1.
\item Merge and center cells A1 to E1.
\item Format cell A1 with 18 pt boldface Arial font family and red colour.
\item Format the range B2:E23 with 12 pt Calibri font family and one decimal place. 
\item Format range B2:E2 with white boldface font and green background colour. 
\item Apply a 2-Color Scale conditional format to the Opening values (range B3:B23).
\item Apply an Arrow Icon Set (red, yellow and green) conditional format to the Maximum values (range C3:C23).
\item Apply a red colour to Minimum values (range D3:D23) under $10,000$ points.
\item Apply a green colour to Closing values above the average.
\item Adjust the width of columns A to E.
\item Save the workbook. 
\end{enumerate}


\end{enumerate}